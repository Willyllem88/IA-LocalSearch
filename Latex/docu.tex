\documentclass[a4paper]{article}
\usepackage[a4paper,left=3cm,right=2cm,top=2.5cm,bottom=2.5cm]{geometry}
\usepackage[utf8]{inputenc}
\usepackage{amsmath}
\usepackage{hyperref}
\usepackage{graphicx}
\graphicspath{ {./images/} }

\title{\textbf{Intel·ligència Artificial:\\
		Pràctica de Cerca Local}}
\author{\emph{Guillem Cabré, Carla Cordero, Hannah Röber}}
\date{Curs 2024-25, Quatrimestre de tardor}

\renewcommand*\contentsname{Continguts}
\renewcommand{\figurename}{Figura}

\begin{document}
	
	\begin{titlepage}
		\clearpage\maketitle
		\thispagestyle{empty}
	\end{titlepage}
	
	\tableofcontents
	\clearpage
	
	\section{Part Descriptiva}
	
	\subsection{Descripció del problema}
	
	La companyia fictícia \emph{Ázamon} ha d'optimitzar els seus enviaments diaris de $n$ paquets a una ciutat, considerant diversos factors. Cada paquet té un pes $w_i$ i una prioritat $p_i$, que defineix el termini màxim per ser entregat. L'empresa rep ofertes diàries de diverses companyies de transport, i el repte és trobar la millor manera de distribuir els paquets entre aquestes ofertes, minimitzant els costos i maximitzant la satisfacció dels clients. \\
	
	Els costos inclouen tant el transport, amb preus per quilogram que varien segons l'empresa, com l'emmagatzematge dels paquets que no es recullen immediatament. La felicitat dels clients augmenta si els paquets arriben abans de la data límit prevista. Per tant, cal equilibrar l'eficiència en costos amb la rapidesa en el lliurament per maximitzar la satisfacció del client. \\
	
	\subsection{Elements del problema}
	
	Cada paquet té un pes $w_i \in \{0.5, 1.0, 1.5, ..., 10.0\}$ kg i una prioritat $p_i \in \{1, 2, 3\}$, que defineix el termini d'entrega: un dia per $p_i = 1$, entre 2 i 3 dies per $p_i = 2$, i entre 4 i 5 dies per $p_i = 3$. Cadascuna d'aquestes prioritats té un cost diferent, $p_i = 1$ val 5 euros, $p_i = 2$ val 3 euros i $p_i = 3$ val 1.5 euros. \\
	
	Les empreses de transport ofereixen cada dia $m$ opcions amb una capacitat màxima $C_j \in \{5, 10, 15, ..., 50\}$ kg, un preu per quilogram transportat $c_j$, i un temps d'entrega $t_j \in \{1, 2, 3, 4, 5\}$ dies. A més, si els paquets no es recullen immediatament, cal assumir un cost d'emmagatzematge de 0.25 euros per quilogram i dia. Els clients es mostren més satisfets amb entregues anticipades, la qual cosa augmenta proporcionalment als dies d'antelació. \\

	\subsection{Definició solució}

	Una solució segons el problema que se'ns ha descrit és una assignació dels paquets a les ofertes de transport del dia apropiades amb les restriccions que la suma dels pesos dels paquets assignats a una oferta no poden superar la seva capacitat màxima i que tots els paquets arribin dins del termini d'entrega que depén de la seva prioritat. \\

	Formalment, si tenim $n$ paquets i $m$ ofertes de transport, una solució seria una tupla que associa cada paquet $p_i$ a una oferta $o_j$, tal que \( 1 \leq i \leq n \) i \( 1 \leq j \leq m \), amb les restriccions de capacitat i temps respectades. \\

	Aleshores el nostre objectiu serà, mitjançant algorismes, generar una solució que compleixi els requisits mencionats, que serà l'estat final, partint d'un estat inicial que ens permeti arribar a una solució més òptima, en la qual es té en compte criteris de qualitat. En tot moment, també s'intentarà maximitzar la felicitat dels clients. \\

	\subsection{Espai de cerca}

	L'espai de cerca dels nostres algorismes és l'espai de solucions al nostre problema, és a dir, la totalitat de les solucions que compleixen els requisits del nostre problema que s'han esmentat en l'apartat anterior sense importar la seva qualitat donada per uns criteris en formen part. És important conèixer la mida de l'espai de cerca per justificar la utilització de certs algorismes per aquest problema en comptes d'utilitzar un algorisme de força bruta senzill. \\

	La grandària de l'espai de cerca està determinat pel nombre de combinacions possibles d'assignació de paquets a ofertes de transport. Aleshores, assumim que $m$ és el nombre d'ofertes del dia que hi ha i que tenim $n$ paquets a distribuir amb la seva prioritat. La grandària màxima de l'espai de cerca seria de l'ordre de O($m^n$), la qual cosa representa totes les combinacions possibles d'assignació de paquets que correspon al cas pitjor on totes les combinacions són vàlides, ja que cada paquet pot ser assignat a qualsevol de les $m$ ofertes. \\

	No obstant això, aquest número pot reduir-se significativament a causa de les restriccions que impedeixen unes certes assignacions inviables, com assignar paquets a una oferta que no pot transportar el pes total o assignar paquets d'alta prioritat a una oferta de llarg temps de lliurament. Aquestes restriccions limiten el nombre de combinacions vàlides, però encara així, l'espai de cerca pot ser considerablement gran, especialment quan el nombre de paquets i ofertes augmenta ja que creixerà  exponencialment. Aleshores, es complica l'eficiència de l'algorisme de força bruta i, per tant, s'ha d'utilitzar altres algorismes. \\

	\subsection{Metodologia de resolució}
	
	Per resoldre aquest problema, utilitzarem algorismes de cerca local. En particular, s'han seleccionat els algorismes de \emph{Hill Climbing} i \emph{Simulated Annealing}, que exploraran l'espai de cerca format per totes les assignacions possibles dels paquets a les ofertes de transport.
	
	\begin{itemize}
		\item L'algorisme de \emph{Hill Climbing} intentarà millorar successivament la solució actual fent petits canvis a l'assignació dels paquets.
		\item L'algorisme de \emph{Simulated Annealing} permetrà l'acceptació temporal de solucions pitjors, amb l'objectiu d'evitar quedar-se atrapats en òptims locals.
	\end{itemize}
	
	A través de diversos experiments, es provaran diferents configuracions de paràmetres i es compararan els resultats dels dos algorismes per determinar quin proporciona millors solucions en termes de cost i felicitat.
	
	\subsection{Implementació de l'Estat}
	
	L'estat del nostre problema s'implementa mitjançant la classe \emph{Estado}, que gestiona l'assignació dels paquets a les diferents ofertes de transport. Aquest estat inclou les següents estructures de dades:
	
	\begin{itemize}
		\item \emph{Paquetes paquetes}: Conté la llista de paquets que s'han de distribuir.
		\item \emph{Transporte ofertas}: Una llista que guarda les ofertes de transport disponibles.
		\item \emph{List$<$Integer$>$ asignaciones}: Una llista que representa les assignacions actuals dels paquets a les ofertes. Cada element de la llista correspon a un paquet. El valor assignat que contenen aquests paquets correspon a l'índex de l'oferta a la qual s'ha assignat (o \(-1\) si no està assignat a cap oferta).
		\item \emph{List$<$Double$>$ espacioDisponibleOfertas}: Guarda l'espai disponible per a cada oferta de transport, que permet controlar la capacitat restant per a cada una de les ofertes assignades.
		\item \emph{Integer felicidad}: Aquest valor guarda la felicitat total acumulada, que s'augmenta quan els paquets s'entreguen abans del que estava previst.
	\end{itemize}
		
	Inicialment, l'estat es configura assignant \(-1\) a cada paquet de la llista \emph{asignaciones}, indicant que no hi ha cap oferta de transport assignada al paquet. A més, \emph{espacioDisponibleOfertas} s'inicialitza amb la capacitat màxima de cada oferta, que posteriorment s'anirà actualitzant quan els paquets es vagin assignant.\\
	
	\subsection{Operadors}
	
	Els operadors pels algoritmes de cerca local seran els següents:
	
	\begin{itemize}
		\item \emph{swapPaquet(p1, p2)}: Siguin $p1$ i $p2$ dos paquets que pertanyen a ofertes de transport diferents, mourà $p1$ a la posició de $p2$ i a l'inrevés. Aquesta operació es fara si i només si cap de les dues ofertes sobrepassa la seva capcacitat màxima.
		\item \emph{mourePaquet(p, o)}: Sigi $p$ un paquet i $o$ una oferta, mourà el paquet $p$ a la oferta $o$. Sempre i quant $p$ càpiga dins de $o$.
	\end{itemize}
	
	Amb aquests dos operadors serem capaços de recòrrer tot l'espai de solucions.
	
	
	\subsection{Generador de Solucions Inicials}
	
	El fet de tenir dues funcions generadores de solucions inicials ens permetrà analitzar el comportament dels dos algorismes de cerca local que utilitzarem. Veurem com la configuració de l'estat inicial ens permetrà, o no, fer una búsqueda eficaç a través de l'espai de cerca i si ens estancarem, o no, en mínims locals. I sobretot, quina de les dues funcions generadores de solucions inicials ens dona un millor rendiment per cadascun dels algorismes. \\
	
	\subsubsection{Entrega just al dia}
	
	Aquesta funció no té en compte la maximització de la felicitat dels clients ni la reducció de costos. Simplement, assigna els paquets a una oferta que compleixi els requisits d'entrega. La idea principal és buscar una oferta que compleixi amb aquesta condició i tingui espai suficient per al paquet. Per evitar quedar-nos sense espai a les ofertes de major prioritat, endrecem la llista de paquets segons aquesta prioritat. D'aquesta manera, assignem abans els paquets més urgents. Per tant, es pot resumir així: \\
	
	\begin{verbatim}
		Ordenar paquets per ordre de prioritat
		Per a cada paquet p_i que pertany a Paquets:
		    j = 0
		    Mentre j < nombre d'ofertes i p_i no estigui assignat:
		        # C_j és la capacitat de la oferta
		        Si p_i <= C_j i i dia_entrega(p_i) <= dia_oferta(o_j):
		            Assignar p_i a o_j
		        Altrament:
		            Incrementar j  # Provar amb la següent oferta o_{j+1}
	\end{verbatim}
	
	
	\subsubsection{Maximitzar la felicitat i minimitzar els costos}
	
	El segon mètode intenta generar una solució inicial que no només respecti els terminis d'entrega, sinó que també busqui maximitzar la felicitat dels clients i minimitzar els costos. \\
	
	Per maximitzar la felicitat haurem d'aconseguir assignar els paquets per tal que s'entreguin tots tan aviat com es pugui. Per fer això, ordenarem la llista d'ofertes de menor a major segons el dia d'entrega ($Ofertes$). Farem el mateix per les prioritats del paquet ($Paquets$). I el mateix per minimitzar els costos: endreçarem primer les ofertes més barates. D'aquesta manera, s'enviaran els paquets el més aviat possible i, en cas d'empat, amb les ofertes de menor cost. \\
	
	Un cop ambdues llistes estan ordenades intentarem assignar els paquets de la següent manera. Vegeu a continuació el pseudocodi: \\
	
	\begin{verbatim}
		Ordenar paquets per ordre de prioritat i per cost
		Ordenar ofertes per dies d'entrega i cost
		Per a cada paquet p_i que pertany a Paquets:
		    j = 0
		    Mentre j < nombre d'ofertes i p_i no estigui assignat:
		        # C_j és la capacitat de la oferta o_j
		        Si p_i <= C_j:
		            Assignar p_i a o_j
		        Altrament:
		            Incrementar j  # Provar amb la següent oferta o_{j+1}
	\end{verbatim}
		
	\subsection{Funció Heurística}
	
	Hem creat dues funcions heurístiques per poder comparar els resultats que ens generen. Cada una d'elles tindrà criteris diferents:
	
	\begin{itemize}
		\item \emph{Heurístic 1}: $H_1 = \text{cost}^2$, aquest heurístic està dissenyat per minimitzar exclusivament els costos, sense tenir en compte ningún altre factor de qualitat. La funció d'evaluació es defineix com el quadrat del cost. Aquest heurístic ens permetrà minimitzar el cost de la assignació de paquets.
		\item \emph{Heurístic 2}: $H_2 = -\text{felicitat}^2 + \text{cost}^2$, per altra banda, aquest heurístic ens permetrà tenir en compte la felicitat total de la assignació de paquets. La funció d'evaluació serà igual que la anterior però li resterem el quadrat de la  felicitat. Així doncs estarem maximitzant felicitat i minimitzant cost a la vegada, ja que volem minimitzar la funció heurística.
	\end{itemize}
	
	\subsection{Resultats}
	
	\newpage
	\section{Part Experimental}
	
	\subsection{Experiment 1}
	
	\subsection{Experiment 2}
	
	\subsection{Experiment 3}
	
	\newpage
	\section{Conclusions}
	
	
	\newpage
	\section{Treball d'Innovació: DeepVariant}
	
	\subsection{Tema}
	Hem escollit DeepVariant, un algorisme de machine learning desenvolupat per Google, que utilitza xarxes neuronals profundes per identificar variants genètiques en dades de seqüenciació de l'ADN. Mitjançant imatges generades a partir de les lectures de seqüència, DeepVariant detecta mutacions petites amb una precisió superior respecte a altres mètodes tradicionals.
	
	\subsection{Responsabilitats d'equip}
		Malgrat que el treball s'està fent col·laborativament i tots els membres del grup participen en totes les tasques, s'ha decidit dividir-ho en tres parts per cercar i aconseguir informació al respecte.
	\begin{itemize}
		\item \emph{Introducció i context}:\\
			Introducció a DeepVariant: què és i per què és rellevant.\\
			Importància de DeepVariant en la detecció de variants genètiques.\\
			Visió general del problema que resol en l'anàlisi de dades de seqüenciació d'ADN.\\
			Per què l'aprenentatge profund és útil en aquest context: beneficis respecte als mètodes tradicionals de detecció de variants (comparació amb altres mètodes).\\
			
		\item \emph{Xarxes Neuronals Convolucionals (CNN)}:\\
			Explicació de què són les xarxes neuronals convolucionals (CNNs).\\
			Com s'utilitzen les CNNs en DeepVariant per identificar patrons genètics.\\
			Avantatges de l'ús de CNNs per analitzar dades de seqüenciació d'ADN (perquè reconeixen patrons visuals).\\
			Relació entre les imatges generades per DeepVariant i l'anàlisi de dades genètiques.\\
			
		\item \emph{Funcionament Pipeline}:\\
			Generació d'imatges a partir de dades de seqüenciació d'ADN.\\
			Com les imatges es passen per la xarxa neuronal convolucional per ser classificades.\\
			Processos d'inferència: com es decideix quina variant està present (substitució, inserció, deleció).\\
			Entrenament de la xarxa neuronal amb dades genètiques.\\
			Presa de decisions sobre la presència de variants genètiques i avantatges de precisió i fiabilitat.\\
	\end{itemize}
	
	
	
	\subsection{Fonts}
	\begin{itemize}
		\item \emph{Creating a universal SNP and small indel variant caller with deep neural networks}:
		Rellevància: Base teòrica de DeepVariant i explicació de l'ús de xarxes neuronales.
		Data d'accés: 13 d'octubre de 2024
		Enllaç: https://doi.org/10.1101/092890
		***********SEGONA PART AL WHATS
		\item \emph{}:
		Enllaç: https://github.com/google/deepvariant
		També porta a altres documents com: https://google.github.io/deepvariant/posts/2020-02-20-looking-through-deepvariants-eyes/
	\end{itemize}
	
	\subsection{Dificultats}
	\begin{itemize}
		\item \emph{Accés a referències}: Alguns articles rellevants requereixen subscripcions per poder obtenir informació detallada.
		\item \emph{Informació tècnica}: Hem hagut de fer un sobreesforç perquè es requereix d'un mínim de coneixements sobre l'ADN per entendre la proposta.
	\end{itemize}

	
	
	

\end{document}